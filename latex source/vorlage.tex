\input{src/header}

\newcommand{\dozent}{Lutz Prechelt}
\newcommand{\tutor}{Samuel Domiks}
\newcommand{\tutoriumNo}{02\\Materialien: Latex, Skript}
\newcommand{\ubungNo}{10}
\newcommand{\veranstaltung}{Softwaretechnik}
\newcommand{\semester}{SoSe21}
\newcommand{\studenten}{Jonny Lam \& Thore Brehmer}

% /////////////////////// BEGIN DOKUMENT /////////////////////////
\begin{document}
\input{src/titlepage}

% /////////////////////// Task 1 /////////////////////////
\section{Markov Generator}
\begin{enumerate}[(a)]
    % /////////////////////// a /////////////////////////
    \item {\itshape Construct a first-order Markov model for this sample.}
    %\includegraphics[width=1\textwidth]{src/u8/task2/8_2.png} 
    %\lstinputlisting[language=python, linerange={15-19}, firstnumber = 15]{src/u7/commands_client.py}  

    % /////////////////////// b /////////////////////////
    \item {\itshape Construct a second-order Markov model for this sample}
   

    % /////////////////////// c /////////////////////////
    \item {\itshape Give one 4-digit PIN number that is generated by your first-order Markov model but not by your second-order model and calculate it’s probability.}
   
    
\end{enumerate}

% /////////////////////// Task 2 /////////////////////////
\section{Project}
\begin{enumerate}[(a)]
    % /////////////////////// a /////////////////////////
    \item {\itshape A client can create a new user and chose a password. You do not have to implement password reset or changing the password..}
    
    % /////////////////////// b /////////////////////////
    \item {\itshape A user can login with a given username and password.}


    % /////////////////////// c /////////////////////////
    \item {\itshape Only authenticated clients can send or receive messages.}

    % /////////////////////// d /////////////////////////
    \item {\itshape The password-information are stored in a file.}

\end{enumerate}

\end{document}